%Page setup
\documentclass[11pt, a4paper]{report}

\usepackage{geometry}
\geometry{a4paper, portrait, top=2cm,bottom=2cm, left=2cm, right=2cm}

%Sort out overly-generous chapter spacing
\usepackage{titlesec}
\titleformat{\chapter}[hang]
{\normalfont\huge\bfseries}{ \thechapter}{10pt}{\Huge}
% Was:[display]{\normalfont\huge\bfseries}{\chaptertitlename\ \thechapter}{20pt}{\Huge}
\titleformat{\section}
{\normalfont\Large\bfseries}{\thesection}{1em}{}
\titleformat{\subsection}
{\normalfont\large\bfseries}{\thesubsection}{1em}{}
\titleformat{\subsubsection}
{\normalfont\normalsize\bfseries}{\thesubsubsection}{1em}{}
\titleformat{\paragraph}[runin]
{\normalfont\normalsize\bfseries}{\theparagraph}{1em}{}
\titleformat{\subparagraph}[runin]
{\normalfont\normalsize\bfseries}{\thesubparagraph}{1em}{}

\titlespacing*{\chapter} {0pt}{-20pt}{10pt} %Was: {0pt}{50pt}{40pt}. Was -20pt for new page
\titlespacing*{\section} {0pt}{3.5ex plus 1ex minus .2ex}{2.3ex plus .2ex}
\titlespacing*{\subsection} {0pt}{3.25ex plus 1ex minus .2ex}{1.5ex plus .2ex}
\titlespacing*{\subsubsection}{0pt}{3.25ex plus 1ex minus .2ex}{1.5ex plus .2ex}
\titlespacing*{\paragraph} {0pt}{3.25ex plus 1ex minus .2ex}{1em}
\titlespacing*{\subparagraph} {\parindent}{3.25ex plus 1ex minus .2ex}{1em}
%\titlespacing{command}{left spacing}{before spacing}{after spacing}[right]
% spacing: how to read {12pt plus 4pt minus 2pt}
%           12pt is what we would like the spacing to be
%           plus 4pt means that TeX can stretch it by at most 4pt
%           minus 2pt means that TeX can shrink it by at most 2pt
%       This is one example of the concept of, 'glue', in TeX

%Remove chapter space in list of figures and tables
\usepackage{etoolbox}% http://ctan.org/pkg/etoolbox
\makeatletter
% \patchcmd{<cmd>}{<search>}{<replace>}{<succes>}{<failure>}
\patchcmd{\@chapter}{\addtocontents{lof}{\protect\addvspace{10\p@}}}{}{}{}% LoF
\patchcmd{\@chapter}{\addtocontents{lot}{\protect\addvspace{10\p@}}}{}{}{}% LoT
\makeatother

%Remove new page break for chapters
%\makeatletter
%\patchcmd{\chapter}{\if@openright\cleardoublepage\else\clearpage\fi}{}{}{}
%\makeatother

%Spec says 11pt Arial;
\usepackage{fontspec}
\setmainfont{Arial}
%\usepackage{mathspec}
%\setmathsfont(Digits)[Numbers={Lining,Proportional}]{Arial}
%\setmathsfont(Latin)[Numbers={Lining,Proportional}]{Arial}
%\setmathsfont(Greek)[Numbers={Lining,Proportional}]{Arial}

%\usepackage[math-style=upright]{unicode-math}
%\setmathfont{Arial}
%\setmathfont[range={\sqrt,\int}]{XITS Math}

%For checking font size
\makeatletter
\newcommand\thefontsize[1]{{#1 The current font size is: \f@size pt\par}}
\makeatother



% Natbib references
\usepackage[numbers]{natbib}

% Hyperref:
% This package makes all references within your document clickable. By default, these references will become boxed and colored. This is turned back to normal with the \hypersetup command below.
\usepackage[pdfpagelabels]{hyperref} %pdfpagelabels allows page numbers such as i and ii in pdf reader
\hypersetup{colorlinks=false,pdfborder=0 0 0}

\usepackage{etoolbox} %Let's let bibliography be ragged right to avoid typesetting problems
\apptocmd{\thebibliography}{\raggedright}{}{}

%Colour
\usepackage[dvipsnames, table]{xcolor}

%Appendix Handling
\usepackage[titletoc,title]{appendix}

%Acronyms!
\usepackage[nonumberlist, acronyms, toc,]{glossaries}
\setacronymstyle{long-short}
\renewcommand{\glsgroupskip}{} %Don't put a small space between groups
\newacronym{matlab}{MATLAB}{matrix laboratory}
\newacronym{pca}{PCA}{priniciple component analysis} %Import list of acronyms

%Display long multiplication
\usepackage{xlop}

%Mathematical symbols
\usepackage{amssymb}
\usepackage{amsmath}
\DeclareMathOperator{\atantwo}{atan2}

%Posh enumeration
\usepackage{enumitem}
\setlist[enumerate]{nosep} %No special separation for enumerated lists
\setlist[itemize]{nosep}


%SI representations
\usepackage{siunitx}

%Drawing
\usepackage{tikz}
\usetikzlibrary{shapes.geometric, shapes.misc, arrows}
\usetikzlibrary{calc}
\usetikzlibrary{positioning}
\usetikzlibrary{patterns}
\usetikzlibrary{automata}
\usepackage{tikz-3dplot}

%Pretty plots
\usepackage{pgfplots}
\pgfplotsset{compat=newest}
% We have some pretty big data sets, so lets use the externalise feature
% See http://tex.stackexchange.com/questions/7953/how-to-expand-texs-main-memory-size-pgfplots-memory-overload
%\usepgfplotslibrary{external} 
%\tikzexternalize[shell escape=-enable-write18, up to date check=diff]

% Pretty tables 
\usepackage{booktabs}

% Caption. For better looking captions the captions.
\usepackage{caption}
\usepackage{subcaption}

%Timing diagrams
\usepackage{tikz-timing}[2011/01/09]

%Listings
\usepackage{listings,chngcntr}
% set the default code style
\lstset{
	frame=tb, % draw a frame at the top and bottom of the code block
	tabsize=4, % tab space width
	breaklines=true, %Wrap text
	showstringspaces=false, % don't mark spaces in strings
	numbers=left, % display line numbers on the left
	commentstyle=\color{ForestGreen}, % comment color
	keywordstyle=\color{blue}, % keyword color
	stringstyle=\color{red}, % string color
	basicstyle=\small %normal font size
}
%Make heading say "list of listings", not just "listings"
\renewcommand\lstlistlistingname{List of Listings}


%Allow easy typesetting of SI units
\usepackage{siunitx}

%Allow floatbarrier
\usepackage{placeins}


%Tables from csv
\usepackage{pgfplotstable}


%Custom commands

%Create new P-R plot
\newcommand{\prplot}[2]
{
	\pgfplotsset{every axis/.append style={
			line width=1pt,
			tick style={line width=0.8pt}}}
	\begin{axis}[
	xlabel=Recall,
	ylabel=Precision,
	%
	ymin =0, ymax = 1.05,
	xmin =0, xmax = 1.05,
	axis lines=left, % Don't put lines on right and top
	%
	width=0.9\columnwidth,
	height = 0.6\columnwidth,
	%
	xtick={0,0.1,...,1.1},
	xmajorgrids=true,
	ytick={0,0.1,...,1.1},
	ymajorgrids=true,
	cycle list name=exotic,
	legend style ={anchor=north east,legend cell align=left,at={(1,1)}},
	mark size=1.75pt,
	]
	
	%Note: pure chance precision will be for  plot 1 only
	\addplot+[mark=none]
	table[ x=recall, y=chance_precision, col sep=comma] {#1};
	\addlegendentry{\shortstack[l]{Chance}};
	
	\addplot
	table[ x=recall, y=precision, col sep=comma] {#1};
	\addlegendentry{#2};
	
	

}

%Add series to P-R plot
\newcommand{\prplotadd}[2]
{
	\addplot
	table[ x=recall, y=precision, col sep=comma] {#1};
	\addlegendentry{#2};
}

%Finish P-R Plot
\newcommand{\prplotclose}
{
	\end{axis}	
}